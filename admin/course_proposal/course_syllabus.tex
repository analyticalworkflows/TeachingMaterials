\documentclass[10pt]{article}
\usepackage[margin=1in]{geometry}
\usepackage{graphicx} 
\usepackage[latin1]{inputenc}
\usepackage{amsmath,amsfonts,amssymb}
\usepackage{enumitem}
\usepackage{hyperref}
\hypersetup{
	colorlinks=true,
	linkcolor=blue,
	filecolor=magenta,      
	urlcolor=blue,
}

\usepackage{titlesec}
\titlespacing\section{0pt}{12pt plus 4pt minus 2pt}{0pt plus 2pt minus 2pt}


\usepackage{eso-pic}

\setlength\parindent{0pt}
\setlength{\leftskip}{1cm}


\begin{document}


\AddToShipoutPicture*
{\put(350,700){\includegraphics[width=6cm]{OSU_horizontal_2C_O_over_B}}}



\section*{Course Name}
Analytical Workflows

\section*{Course Number}
 IB 516

\section*{Course Credits}
4

\section*{Course Meeting Times}
Lecture, Tuesday \& Thursday 10:00-11:50 am

\section*{Prerequisite and/or Corequisite}
Prerequisite: None

\section*{Course Catalog Description}
Examines and implements the theory and implementation of efficient, reproducible workflows -- 
including best practices in scientific programming, project management, and collaboration --  for 
computational, analytical, and data-driven biological research.

\section*{Course Contents}
Please see the Schedule of topics on the Github repository here: \url{https://github.com/analyticalworkflows/TeachingMaterials}. It is in the ``Schedule'' section of the file README.md, which you can view by scrolling down on the webpage.

\section*{Course Specific Measurable Student Learning Outcomes}
After successful completion of this course, students will be able to:
\begin{enumerate}[leftmargin=2cm]
	\itemsep0em
	\item Translate a research plan into an explicit analytical workflow;
	\item Apply best practices in scientific programming to construct reproducible research;
	\item Manage and collaborate on complex research projects using a version control system;
	\item Apply analytical workflows to advance their dissertation research.
\end{enumerate}

\clearpage
\section*{Evaluation of Student Performance}
Our primary goal in this course is for students to develop more efficient research skills.  
An important secondary goal is to have students make significant progress on their thesis work.  
Our philosophy is that students can achieve both because our primary goal is best achieved by having 
students practice new tools while working on their own research.
\\\\
Student learning will be verified during weekly ``hack-a-thon'' sessions, by weekly assessment of their 
``pushes'' of ``commits'' to their GitHub repositories, and by evaluation of three project presentations.  
(See rubrics at 
\url{https://github.com/analyticalworkflows/TeachingMaterials/tree/master/course_info/rubrics}).

\begin{itemize}[leftmargin=2cm]
\itemsep0em
\item Hack-a-thon \& Paper discussions \emph{- 200 points}
\item GitHub commits \emph{- 200 points}
\item GitHub README Markdown page \emph{- 50 points}
\item \LaTeX\ summary report \emph{- 50 points}
\item Project proposal presentation \emph{- 200 points}
\item Project progress presentation \emph{- 200 points}
\item Project final report presentation \emph{- 300 points}
\item Total \emph{- 1200 points}
\end{itemize}

\begin{center}
\begin{tabular}{|c|c||c|c|}
	\hline
	Grade &  Percent Range & Grade  & Percent Range   \\
	\hline
	\hline
	A & 94-100 &  C &  73-75 \\
	\hline
	A-&  90-93&  C-&  70-72  \\
	\hline
	B+& 86-89 &  D+& 66-69 \\
	\hline
	B & 83-85 &  D&  63-65 \\
	\hline
	B- & 80-82 & D- &  60-62 \\
	\hline
	C+&  76-79&  F & $< 60$ \\
	\hline
	\hline
\end{tabular}
\end{center}



\section*{Learning Resources}
All learning resources will be supplied at no cost. 
All used software will be free and open-source.\\ 
For complete access to all teaching materials and learning resources, see\\
\url{https://github.com/analyticalworkflows/TeachingMaterials}.

\clearpage

\section*{Course Policies }


\subsection*{Academic Calendar}
All students are subject to the registration and refund deadlines as stated in the Academic Calendar: 
\url{https://registrar.oregonstate.edu/osu-academic-calendar}

\subsection*{Statement Regarding Students with Disabilities}
Accommodations for students with disabilities are determined and approved by Disability Access 
Services (DAS). 
If you, as a student, believe you are eligible for accommodations but have not obtained 
approval please contact DAS immediately at 541-737-4098 or at \url{http://ds.oregonstate.edu}. 
DAS notifies 
students and faculty members of approved academic accommodations and coordinates implementation 
of those accommodations. 
While not required, students and faculty members are encouraged to discuss 
details of the implementation of individual accommodations.

\subsection*{Student Conduct Expectations}
\url{https://beav.es/codeofconduct}

\subsection*{Reach Out for Success}
University students encounter setbacks from time to time. If you encounter difficulties and need 
assistance, it is important to reach out. Consider discussing the situation with an instructor or academic 
advisor.
 Learn about resources that assist with wellness and academic success at 
\url{oregonstate.edu/ReachOut}.
 If you are in immediate crisis, please contact the Crisis Text Line by texting 
OREGON to 741-741 or call the National Suicide Prevention Lifeline at 1-800-273-TALK (8255)

\subsection*{Student Bill of Rights}
OSU has twelve established student rights. They include due process in all university disciplinary processes, an equal opportunity to learn, and grading in accordance with the course syllabus: \url{https://asosu.oregonstate.edu/advocacy/rights}

\subsection*{Student Learning Experience Survey}
During Fall, Winter, and Spring term the online Student Learning Experience surveys open to students the Wednesday of week 9 and close the Sunday before Finals Week. Students will receive notification, instructions, and the link through their ONID email. They may also log into the survey via MyOregonState or directly at \url{https://beav.es/Student-Learning-Survey}. Survey results are extremely important and are used to help improve courses and the learning experience of future students. Responses are anonymous (unless a student chooses to ``sign'' their comments, agreeing to relinquish anonymity of written comments) and are not available to instructors until after grades have been posted. The results of scaled questions and signed comments go to both the instructor and their unit head/supervisor. Anonymous (unsigned) comments go to the instructor only.


\end{document}
