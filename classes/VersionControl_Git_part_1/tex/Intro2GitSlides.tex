\documentclass{beamer}

% \usepackage{beamerthemesplit} // Activate for custom appearance
%Unnumbered footnotes
\newcommand\ufoot[1]{
\begingroup
\renewcommand\thefootnote{}\footnote{#1}
\addtocounter{footnote}{-1}
\endgroup
}



\title{Introduction to Git and Github Part I}
\subtitle{IB 516 Analytical Workflows}
\author{Ben Dalziel}
\date{\today}

\begin{document}

\frame{\titlepage}

\section[Outline]{}
\frame{\tableofcontents}

\section{Introduction to Version Control}

\frame{
	\frametitle{What is a version control system?}

	\begin{itemize}
	\item Records changes in a set of files over time so that you can recall specific versions later.
	\item Allows 'time travel' back and forth from any previous time in your project development.
	\item Allows 'peaceful coexistence and exchange of info between 'parallel universes' of a project'
	\item Allows stable and efficient collaborations (with others and with your past self) that produce reproducible work.
	\end{itemize}
	
}


\frame{
	\frametitle{In practice}
	
	\begin{itemize}
	\item Revert selected files back to a previous state.
	\item Revert the entire project back to a previous state.
	\item Compare changes over time.
	\item See who last modified something that might be causing a problem, who introduced an issue and when, and more. 
	\item If you screw things up or lose files, you can easily recover. 
	\item You get all this for very little effort / overhead (aside from startup costs for learning).
	\end{itemize}
	
	\ufoot{\tiny{https://git-scm.com/book/en/v2/Getting-Started-About-Version-Control}}

}


\frame{
	\frametitle{Visualizing a version controlled project through time}
	
	\includegraphics[scale = 0.3]{figs/pretty_branch_graph}
	
	\ufoot{\tiny{https://stackoverflow.com/a/24107223}}


}

\frame{
	\frametitle{Version control in Git: How Git stores data}
	
	\includegraphics[scale = 0.4]{figs/how_git_stores_data}
	
	\ufoot{\tiny{http://git-scm.com/book/en/v2/Getting-Started-What-is-Git}}
	
	
}




\section{Basic Workflow}


\frame{
	\frametitle{Basic lifecyle of work on a file}
	
	\includegraphics[scale = 0.4]{figs/lifecycle}
		
	\ufoot{\tiny{http://git-scm.com/book/en/v2/Getting-Started-What-is-Git}}
	
	\tiny{*We haven't talked about $staging$ which is an intermediate step between modifying a file and committing it. Git allows you to choose which modified files will be $staged$ (i.e. flagged for inclusion) as part of the next commit. Typically we will want to commit all modifications. Github Desktop automatically stages all modifications, so we don't need to think too much about staging at the moment.}
	
}

\frame{
	\frametitle{Some Git words}
	
	repository, commit, push, fetch, merge, pull, local, origin, master, branch, fork

	
}
	
\frame{
	\frametitle{Naming commits}
	
	\begin{itemize}
		\item Use the 'imperative mood.' %(e.g. ``finish your homework'')
        		\item Complete the sentence If this commit is adopted , it will...
       		\item Capitalize the first word.
		\item Do not use a period.
	\end{itemize}
	
	
	\includegraphics[scale = 0.5]{figs/commit_messages} \\
	
	\ufoot{\tiny{https://chris.beams.io/posts/git-commit/}}
}


\end{document}
