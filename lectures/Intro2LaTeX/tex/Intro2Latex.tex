\documentclass[12pt,letterpaper]{article}
\usepackage[latin1]{inputenc}
\usepackage{amsmath}
\usepackage{amsfonts}
\usepackage{amssymb}
\usepackage{graphicx}

% the above packages are the "base"

\usepackage{hyperref}  % I want this one for the \url{} function.


%  You "comment out" lines with the % symbol.


\author{Mark Novak}
\title{This will be my article title.}


\begin{document}
\maketitle

\tableofcontents % If I use \sections then I can auto-generate a table of contents:

\pagebreak

\section{Introduction } \label{sec:intro} % labels allow automatic referencing

You can do all the following without a \LaTeX gui using \url{https://www.overleaf.com}.  For writing one-off equations use LaTeXiT \url{https://www.chachatelier.fr/latexit/}.  To find a symbol:  \url{http://detexify.kirelabs.org/classify.html}.

\subsection*{My un-numbered subsection}

The fundamental principle of calculus entails
\begin{equation}
	\label{eqn:calculus}
	\lim_{{\Delta x} \to 0}\frac{f(a+\Delta x)-f(a)}{\Delta x}.
\end{equation}

\noindent % if you don't want a particular section to auto-indent
Calculus rocks because it can do all of the following:
\begin{itemize} 
	\item first item
\end{itemize}

One could also organize all the cools things it can do in a table
\begin{center}
\begin{tabular}{l c} % {l c} means the 1st column is left-aligned while the 2nd is centered
	\label{tab:mytable}
		Reason & Explanation\\ % the \\ is  a "return/next line"
		 \hline
		1 & blah blah blah \\ 
 		 \hline
\end{tabular}
\end{center}

I can easily reference the section (sect.\ref{sec:intro}), equation (eqn. \ref{eqn:calculus}) and table (Table \ref{tab:mytable}).  Their numbers will auto-generate, which makes it easy to move them around in your paper and adhere to a journal's stylistic preferences.

\section{Figures}
Note that the position of figures is auto-determined!

% Figures are included like this:
\begin{figure}
	\centering
	\includegraphics[width=0.2\linewidth]{figs/LaTeX_logo.png}
	\caption{This is the \LaTeX logo.}
	\label{fig:logo}
\end{figure}

\end{document}