\documentclass[10pt]{article}
\usepackage[latin1]{inputenc}
\usepackage{amsmath}
\usepackage{amsfonts}
\usepackage{amssymb}
\usepackage{graphicx}
\usepackage[text={6.5in,8.5in},centering]{geometry}
\geometry{verbose,a4paper,tmargin=2.4cm,bmargin=2.4cm,lmargin=2.4cm,rmargin=2.4cm}

\author{Mark Novak \& Ben Dalziel}
\title{Course Proposal:\\Analytical Workflows (IB 5XX)}

\begin{document}

\maketitle
\date{}

\section*{Course motivation}
Graduate students in our department are amassing ever larger datasets to ask ever more challenging questions with ever more sophisticated analytical techniques.   Moreover, to keep pace with their disciplines, they must increasingly straddle empirical, statistical, and theoretical approaches, hence many propose dissertation chapters entailing mathematical models and simulations.  The large-scale, multi-faceted, and ever more collaborative nature of their projects lies at the heart of our department's \emph{integrative} approach to biology.  And yet, no where in our curriculum* are students provided the tools with which to manage this complexity.

Our proposed course will provide students the tools they need to organize and manage their projects from inception to publication.  Working with their own data and models, students will experience and practice: (re)organizing projects into efficient, reproducible, and easily-modified ``\emph{analytical workflows}''; writing modular, easily-debugged code using best-practices in coding grammar and data visualization; tracking and communicating progress and hurdles (issues) to others using stand-alone and cloud-based version control and collaboration software.

\begin{quote}
	*Existing \emph{omics} courses introduce many students to the concept of \emph{analysis pipelines} which are akin to \emph{workflows}, but these courses do not cover issues relating to project management as a whole, are specific to particular programming environments, and are not of relevance to students pursuing non-\emph{omic} research.
\end{quote}


\section*{Student-targeted course advertisement}
Have you proposed a modeling chapter for your dissertation but need support getting things up and running?  Are you sitting on a giant data set ready for analysis and visualization but don't know how or where to begin?  Maybe you're far along in some series of analyses and feel ``lost in the trees.''

This class will help you with these challenges by practicing the development and implementation of efficient, reproducible \emph{workflows} for your projects.  Every project should (and can) be modular and fully automated, hence reproducible, portable and easily modified.  Rerunning a model under a different set of parameters should (and can) be as simple as a few keystrokes. Regenerating all analyses, figures and tables after finding a typo in your code or dataset should (and can) be painless.

Efficient workflows start at project conception and end only if the project idea is itself a dead end.  Thus, in this class, we'll work to practice (1) refining and articulating project ideas and goals, (2) creating modular and automated analyses, and (3) using best-practices in coding and project management. We'll learn how to use Git, GitHub, \LaTeX\, and Markdown.  The instructors will mostly use \textsf{R} within RStudio, but users of other programming languages and text editors are welcome and encouraged.  You will need either (1) a large unwieldy dataset and an end goal (e.g., reproducing someone else's analysis or visualization) or (2) a dynamical model or simulation (or sufficiently well-developed ideas for one).  The use of other people's data or published models is also encouraged, as needed.

\section*{Student testimonials}
A preliminary IB599 version of this course was taught by Ben Dalziel (with support from Mark Novak) in the Spring of 2019.  Student eSET scores for this course were 6.0 (vs. the departmental mean of 4.9) and 6.0 (vs. the departmental mean of 5.1) out of 6.0 points possible for ``the course as a whole'' and ``the instructor's contribution'' respectively. The following are excerpts of the feedback that was received via email and the eSET evaluations.

\begin{quote}
\emph{``Ben and Mark helped turn a daunting task that I'd been putting off (my modeling chapter) into a well-organized reality! Five stars.''}
\end{quote}
\begin{quote}
	\emph{``I wasn't experienced in R/Git or savvy about best practices in coding or reproducible workflows. Ben and Mark helped me structure my independent work and provided essential information about programs and best practices in coding.''}
\end{quote}
\begin{quote}
	\emph{``After taking the class, [...] I feel empowered to solve issues with my model on my own.''}
\end{quote}
\begin{quote}
	\emph{``I really appreciate that you shared your experience and learning process with us (metacognitive reflection! best teaching practices!).''}
\end{quote}
\begin{quote}
	\emph{``I'm so happy this class exists, and it was instrumental to much of the progress I've made this term.''}
\end{quote}
\begin{quote}
	\emph{``I am of the opinion that this Analytical Workflow class is indispensable. Engaging in research comes with a massive deluge of papers, files, data, output, etc, and yet prior to this class I had never before been exposed to organizational ``best practices'' recommendations.''}
\end{quote}
\begin{quote}
	\emph{``This class has been extremely useful for the conceptual organization and technical execution of my research. Thank you for organizing and leading this class!''}
\end{quote}



\section*{Course details}
\emph{Credits:} 4\\
\emph{Quarter:} Winter\\
\emph{Course times:} Tuesday \& Thursday 10:00-11:50\\
\emph{Frequency:} Annually (starting 2021)\\
\emph{Instructor of record:} Mark Novak, Ben Dalziel (alternating annually)\\
\emph{Prerequisites}: Graduate standing, or by instructor permission\\

\end{document}